
\documentclass[10]{article}
\usepackage{graphicx}
\begin{document}
\title{Step 2 - Work Log \\
M2M Lectures\\
Grenoble University}
\author{Your Names Here}
\date{\today}
\maketitle

\section{Preface by Pr. Olivier Gruber}

This document is your work log for the second step in the 
M2M course, leading to build your own mini-distribution.

Like for the first step, this work log has two parts. 
One part is about diverse sections, that you must fill up.
Each section has a bunch of questions to help you do that.
The questions provide a guideline for your learning. 
They are about giving your pointers on what to learn about.
The other part is really a laboratory log, 
keeping track of what you do, as you do.

{\bf REMEMBER:} plaggia is a crime that can get you evicted
forever from french universities... The solution is simple,
write using your own words or quote, giving the source of
the quoted text. Also, remember that you do not learn through
cut\&paste. You also do not learn much by watching somebody else
doing.

\section{Outline of Step2}

To build your own distribution, almost from scratch, you will need
to acquire a number of skills:

\begin{itemize}
\item
GRUB bootloader
\item
Linux kernel
\item
Distribution layout and core files
\end{itemize}

Many pieces are provided to you, pieces that you are asked to understand.
First, you are given a compiled, ready-to-use, boot loader (GRUB 0.97).
Second, you are given a minimal distribution, with a linux kernel.

You can build a Qemu disk with mkdist.sh sript, following the on-screen
instructions. Once you have a disk, you can boot Qemu with it.
When booting, you will get a GRUB menu, with several boot options,
one is called ``Hello - RootFS''.
Select that option and you will get something very similar 
to what you did in step1, but using a real bootloader and a linux kernel.

After that, the learning process will start and it cannot be done entirely 
as a linear process, sorry. So you will fill up the various sections
incrementally, certainly in an order that will be a bit erratic.

We suggest the following steps.

\begin{enumerate}
\item Bootable disk and GRUB
\begin{itemize}
\item Read and start understanding the script: mkdist.sh
Explain each step, explaining the related concepts such as 
the loopback mount, mknod files (character and block).
\item Read about and understand the GRUB process and how it is
installed on a disk and the GRUB boot stages.
\item Read about and understand the GRUB menu given to you.
\end{itemize}

\item Hello - RootFS
\begin{itemize}
\item Understand the first boot option of the GRUB menu (Hello - RootFS).
Which kernel boots? What is the role of initrd.hello? Look at the
hello directory and how hello is compiled, linked, and the initrd.hello
is constructed. Explain how it works and why.
\item Read about and understand the purpose of initial ramdisks (initrd)
for booting linux in general, on any machine.
\end{itemize}

\item Hello - Disk
\begin{itemize}
\item Read and understand the second boot option (Hello - Disk).
What is different in the boot process? Explain how this process
boots the linux kernel from the disk you created with mkdist.sh.
\item Explain the two console options given to the kernel?
Why are they useful? How do you exploit them when booting from Qemu?
\item Why do we use a different kernel for this boot process?
Instead of using the generic vmlinuz-2.6.31-22?
\item Compare the different kernel configurations given to you:
allnoconfig-2.6.32.65, miniconfig-2.6.32.65, m2m-config-2.6.32.65.
(hint: compiling a linux kernel, make allnoconfig, make bzImage)
(hint2: look at Section~\ref{sec:linux:kernel:config})

\end{itemize}

\item MiniDist Layout
\begin{itemize}
\item Dive into the given MiniDist and explain its layout,
giving the purpose of the different directories.
\item Explain the purpose of the few files in those directories.
In particular the various files under /dev.
\item What is missing for a realistic minidist?
(hint: libraries, commands, shell, configuration files and scripts).
\end{itemize}

\end{enumerate}

Here are a bunch of questions and points to help you progress
and focus your attention.

\begin{enumerate}
\item
Layout of the Qemu disk, in terms of GRUB stages and file system partition.
\item
Partition table in the MBR (see parted and fdisk)
\item 
What kind of file system are we using.
Why only ext2? Could we use something other format?
What do we have to pay attention to? Think GRUB
and Linux kernel.
\item
Look in the GRUB manual how to install GRUB manually.
Explain the manual steps during the mkdist.sh.
Explain why they are necessary. 
\item
Explain the loop-back setup with the Qemu disk.
Why is it necessary? 
\item 
What are those files under /dev?
What is their purpose? How are they created? (hint mkdnod)
\item 
During the mkdist.sh, we boot Qemu twice in a row, 
with a manual GRUB setup in between.
What is going on? 
Explain why we stop at the GRUB prompt the first time
and why we get a GRUB menu the second time around.
\item 
Look at the GRUB menu and understand its structure.
\item 
Look at the linux kernel options in the GRUB menu.
What are they for? Why use those options?
Hint: console history, see the differences between
booting Hello-RootFS and Hello-Disk.
\item
In your mini distribution, under /boot, you have
several kernels. Regarding the 2.6.31.22-generic,
what are the different files related to this version
(System.map, config, and vmlinuz).
\item
Explain how the linux kernel boots with Hello-RootFS.
Hint: initial ramdisk (initrd).
\item
Explain how the hello program can actually execute?
Indeed, the mini distribution has no libraries.
Hint: hello makefile.
\item 
Try replacing the kernel in the Hello-Disk configuration,
from vmlinuz to vmlinuz-2.6.31-22-generic, what is happening?
Why? Why is it working with vmlinuz.

\end{enumerate}



\section{Qemu}

This section is here in case you learn some more stuff about Qemu.
You will find plenty of information in the README-QEMU document.

\section{MiniDist Script}

You ask to read and explain the following script (mkdist.sh).

\section{GRUB} 

In the grub directory, you will have binary and the sources for
grub-0.97, an older and much simpler GRUB version.

You can find some relevant documentations in the docs directory,
in particular the GRUB manual and a great explaination of QEMU disk
images.

\subsection{GRUB principles and stages}

\begin{enumerate}
\item
Look at eltorito stage for CD-ROMs.
\item
Look at the various stages for a disk (stage1, stage1.5, stage2).
\item
Look in the GRUB manual how to install GRUB manually.
\item 
Look at GRUB menu.
\end{enumerate}

\subsubsection{Building GRUB}

{\bf --- THIS SECTION IS FOR YOUR INFORMATION ONLY --- },

{\bf Do not rebuild GRUB}, unless you have to and you are damn sure. 
You have been given a working for x86\_32 (IA-32). 

For your information, be warned that to build this version of GRUB, 
you will need GCC-3.4. So you can install it on your system, via apt-get install.
Let's assume you have downloaded your sources under ~/M2M/grub/grub-0.97
in your dev guest. You can compile and install like this:

{\em\small
\begin{verbatim}
        # export CC=gcc-3.4
	# cd ~/M2M/grub/grub-0.97
	# ./configure –prefix=~/M2M/grub/grub
	# make
	# make install 
\end{verbatim}
}

Notice the important setup in the configure step
where you specify a path where to install GRUB.
Notice this is a local path to your home directory,
{\bf DO NOT INSTALL IT ON YOUR MACHINE.}
By default, it installs on /boot/grub
and this is a really bad idea.

\section{GRUB and Linux Boot Process}

Look at what is going on in the /boot directory.
The idea here is to be able to explain the boot process,
starting with GRUB and ending with the linux kernel booting.

You must understand the three options in the GRUB menu.

\begin{enumerate}
\item
Hello - RootFS
\item
Hello - Disk
\item
Your Mini Distribution
\end{enumerate}

Each represents a different boot strategy.

\section{Linux Kernel Configurations}
\label{sec:linux:kernel:config}

You are given three configurations:

\begin{enumerate}
\item allnoconfig-2.6.32.65
\item miniconfig-2.6.32.65
\item m2m-config-2.6.32.65.
\end{enumerate}

All three for the version 2.6.32.65 of the linux kernel.

The all-no-config is generated with the ``make allnoconfig''
from a kernel source directory. It corresponds to the absolutely
smallest linux kernel, with all optional code turned off,
but it is no longer functional.

The mini-config is a functional kernel, corresponding to the
kernel given in the MiniDist, called vmlinuz. You are asked
to compare and explain the differences between the all-no-config
and the mini-config. 

\noindent (hint: a usefull program for this is called meld.)

\noindent (hint: the differences are related to the minimal
hardware support, like adding a pci bus support).

Then you are asked to do the same between the mini-config
and the m2m-config. The m2m-config is intended for a kernel
specifically compiled for our target distribution: small,
booting directly from disk, and with network connectivity.

You are asked the Qemu configuration that would correspond
to booting this kernel. (hint: look at README-QEMU).


\section{Mini-Distribution}

You are given this layout for your mini-distribution.
{\em\small 
\begin{verbatim}
  MiniDist$ ls
  bin  boot  dev  etc  hello  lib  root  sbin
  MiniDist$
\end{verbatim}
}

Explain each directory and each directory's contents.

If you add anything, you will document what you add and why.

\section{Laboratory Log}

\end{document}
